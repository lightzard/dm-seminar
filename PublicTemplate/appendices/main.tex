\chapter{Appendix}
\begin{lstlisting}[language=Python,label={code:lenet},caption={LeNet code}]
model = Sequential()
model.add(Conv2D(6, (3, 3), activation="tanh", input_shape=(28, 28, 1)))
model.add(MaxPooling2D(pool_size=(2, 2)))
model.add(Conv2D(16, (2, 2), activation="tanh"))
model.add(MaxPooling2D(pool_size=(2, 2)))
model.add(Flatten())
model.add(Dense(256, activation="tanh"))
model.add(Dense(10, activation="softmax"))
model.compile("adadelta", "categorical_crossentropy", metrics=['accuracy'])
model.summary()
model.fit(x_train, y_train, batch_size=128, epochs=30, validation_data=(x_test, y_test))

score = model.evaluate(x_test, y_test, verbose=1)
\end{lstlisting}

\begin{lstlisting}[language=Python,label={code:vggnet},caption={VGG-like Net Code}]
model=Sequential()
model.add(Conv2D(filters=32, kernel_size=(3, 3), padding="same", 
input_shape=x_train2.shape[1:], activation='relu'))
model.add(Conv2D(filters=64, kernel_size=(3, 3), padding="same", activation='relu'))
model.add(MaxPooling2D(pool_size=(2, 2)))
model.add(Dropout(0.5))
model.add(Conv2D(filters=128, kernel_size=(3, 3), padding="same", activation='relu'))
model.add(Conv2D(filters=256, kernel_size=(3, 3), padding="valid", activation='relu'))
model.add(MaxPooling2D(pool_size=(3, 3)))
model.add(Dropout(0.5))
model.add(Flatten())
model.add(Dense(256))
model.add(LeakyReLU())
model.add(Dropout(0.5))
model.add(Dense(256))
model.add(LeakyReLU())
model.add(Dense(10, activation='softmax'))
model.compile(loss='categorical_crossentropy', optimizer='adadelta', metrics=['accuracy'])
model.summary()

model.fit(x_train, y_train, batch_size=128, epochs=10, validation_data=(x_test, y_test))

score = model.evaluate(x_test, y_test, verbose=1)
print('Test score:', score[0])
print('Test accuracy:', score[1])
\end{lstlisting}

\begin{lstlisting}[language=Python,label={code:resnet},caption={ResNet Code}]
#resnet https://github.com/keras-team/keras/blob/master/examples/cifar10_resnet.py

# Training parameters
batch_size = 32  # orig paper trained all networks with batch_size=128
epochs = 10
data_augmentation = True
num_classes = 10

#Adjusting both parameter for using ResNet V1 or V2 and its depth
n = 3
version = 1

# Computed depth from supplied model parameter n
if version == 1:
depth = n * 6 + 2
elif version == 2:
depth = n * 9 + 2

# Model name, depth and version
model_type = 'ResNet%dv%d' % (depth, version)

print('x_train2 shape:', x_train2.shape)
print(x_train2.shape[0], 'train samples')
print(x_test2.shape[0], 'test samples')
print('y_train shape:', y_train2.shape)


def lr_schedule(epoch):
"""Learning Rate Schedule
Learning rate is scheduled to be reduced after 80, 120, 160, 180 epochs.
Called automatically every epoch as part of callbacks during training.
# Arguments
epoch (int): The number of epochs
# Returns
lr (float32): learning rate
"""
lr = 1e-3
if epoch > 180:
lr *= 0.5e-3
elif epoch > 160:
lr *= 1e-3
elif epoch > 120:
lr *= 1e-2
elif epoch > 80:
lr *= 1e-1
print('Learning rate: ', lr)
return lr

def resnet_first_block(inputs,
num_filters=16,
kernel_size=3,
strides=1,
activation='relu',
batch_normalization=True,
conv_first=True):

x = ZeroPadding2D(padding=(2, 2), data_format=None)(inputs)

y = Conv2D(num_filters,
kernel_size=kernel_size,
strides=strides,
padding='same',
kernel_initializer='he_normal',
kernel_regularizer=l2(1e-4))(x)

return y

def resnet_block(inputs,
num_filters=16,
kernel_size=3,
strides=1,
activation='relu',
batch_normalization=True,
conv_first=True):
"""2D Convolution-Batch Normalization-Activation stack builder
# Arguments
inputs (tensor): input tensor from input image or previous layer
num_filters (int): Conv2D number of filters
kernel_size (int): Conv2D square kernel dimensions
strides (int): Conv2D square stride dimensions
activation (string): activation name
batch_normalization (bool): whether to include batch normalization
conv_first (bool): conv-bn-activation (True) or
activation-bn-conv (False)
# Returns
x (tensor): tensor as input to the next layer
"""
x = inputs
if conv_first:
x = Conv2D(num_filters,
kernel_size=kernel_size,
strides=strides,
padding='same',
kernel_initializer='he_normal',
kernel_regularizer=l2(1e-4))(x)
if batch_normalization:
x = BatchNormalization()(x)
if activation:
x = Activation(activation)(x)
return x
if batch_normalization:
x = BatchNormalization()(x)
if activation:
x = Activation('relu')(x)
x = Conv2D(num_filters,
kernel_size=kernel_size,
strides=strides,
padding='same',
kernel_initializer='he_normal',
kernel_regularizer=l2(1e-4))(x)
return x


def resnet_v1(input_shape, depth, num_classes=10):
if (depth - 2) % 6 != 0:
raise ValueError('depth should be 6n+2 (eg 20, 32, 44 in [a])')
# Start model definition.
inputs = Input(shape=input_shape)
num_filters = 16
num_sub_blocks = int((depth - 2) / 6)

x = resnet_first_block(inputs=inputs)
# Instantiate convolutional base (stack of blocks).
for i in range(3):
for j in range(num_sub_blocks):
strides = 1
is_first_layer_but_not_first_block = j == 0 and i > 0
if is_first_layer_but_not_first_block:
strides = 2
y = resnet_block(inputs=x,
num_filters=num_filters,
strides=strides)
y = resnet_block(inputs=y,
num_filters=num_filters,
activation=None)
if is_first_layer_but_not_first_block:
x = resnet_block(inputs=x,
num_filters=num_filters,
kernel_size=1,
strides=strides,
activation=None,
batch_normalization=False)
x = keras.layers.add([x, y])
x = Activation('relu')(x)
num_filters = 2 * num_filters

# Add classifier on top.
# v1 does not use BN after last shortcut connection-ReLU
x = AveragePooling2D(pool_size=8)(x)
y = Flatten()(x)
outputs = Dense(num_classes,
activation='softmax',
kernel_initializer='he_normal')(y)

# Instantiate model.
model = Model(inputs=inputs, outputs=outputs)
return model


def resnet_v2(input_shape, depth, num_classes=10):

if (depth - 2) % 9 != 0:
raise ValueError('depth should be 9n+2 (eg 56 or 110 in [b])')
# Start model definition.
inputs = Input(shape=input_shape)
num_filters_in = 16
num_filters_out = 64
filter_multiplier = 4
num_sub_blocks = int((depth - 2) / 9)

# v2 performs Conv2D with BN-ReLU on input before splitting into 2 paths
x = resnet_block(inputs=inputs,
num_filters=num_filters_in,
conv_first=True)

# Instantiate convolutional base (stack of blocks).
activation = None
batch_normalization = False
for i in range(3):
if i > 0:
filter_multiplier = 2
num_filters_out = num_filters_in * filter_multiplier

for j in range(num_sub_blocks):
strides = 1
is_first_layer_but_not_first_block = j == 0 and i > 0
if is_first_layer_but_not_first_block:
strides = 2
y = resnet_block(inputs=x,
num_filters=num_filters_in,
kernel_size=1,
strides=strides,
activation=activation,
batch_normalization=batch_normalization,
conv_first=False)
activation = 'relu'
batch_normalization = True
y = resnet_block(inputs=y,
num_filters=num_filters_in,
conv_first=False)
y = resnet_block(inputs=y,
num_filters=num_filters_out,
kernel_size=1,
conv_first=False)
if j == 0:
x = resnet_block(inputs=x,
num_filters=num_filters_out,
kernel_size=1,
strides=strides,
activation=None,
batch_normalization=False)
x = keras.layers.add([x, y])

num_filters_in = num_filters_out

# Add classifier on top.
# v2 has BN-ReLU before Pooling
x = BatchNormalization()(x)
x = Activation('relu')(x)
x = AveragePooling2D(pool_size=8)(x)
y = Flatten()(x)
outputs = Dense(num_classes,
activation='softmax',
kernel_initializer='he_normal')(y)

# Instantiate model.
model = Model(inputs=inputs, outputs=outputs)
return model


if version == 2:
model = resnet_v2(input_shape=input_shape, depth=depth)
else:
model = resnet_v1(input_shape=input_shape, depth=depth)

model.compile(loss='categorical_crossentropy',
optimizer=Adam(lr=lr_schedule(0)),
metrics=['accuracy'])
model.summary()
print(model_type)
save_dir = os.path.join(os.getcwd(), 'saved_models')
model_name = 'cifar10_%s_model.{epoch:03d}.h5' % model_type
if not os.path.isdir(save_dir):
os.makedirs(save_dir)
# Prepare model model saving directory.
filepath = os.path.join(save_dir, model_name)
# Prepare callbacks for model saving and for learning rate adjustment.
checkpoint = ModelCheckpoint(filepath=filepath,
monitor='val_acc',
verbose=1,
save_best_only=True)

lr_scheduler = LearningRateScheduler(lr_schedule)

lr_reducer = ReduceLROnPlateau(factor=np.sqrt(0.1),
cooldown=0,
patience=5,
min_lr=0.5e-6)

callbacks = [checkpoint, lr_reducer, lr_scheduler]

# Run training, with or without data augmentation.
if not data_augmentation:
print('Not using data augmentation.')
model.fit(x_train2, y_train2,
batch_size=batch_size,
epochs=epochs,
validation_data=(x_test2, y_test2),
shuffle=True,
callbacks=callbacks)
else:
print('Using real-time data augmentation.')
# This will do preprocessing and realtime data augmentation:
datagen = ImageDataGenerator(
# set input mean to 0 over the dataset
featurewise_center=False,
# set each sample mean to 0
samplewise_center=False,
# divide inputs by std of dataset
featurewise_std_normalization=False,
# divide each input by its std
samplewise_std_normalization=False,
# apply ZCA whitening
zca_whitening=False,
# randomly rotate images in the range (deg 0 to 180)
rotation_range=0,
# randomly shift images horizontally
width_shift_range=0.1,
# randomly shift images vertically
height_shift_range=0.1,
# randomly flip images
horizontal_flip=True,
# randomly flip images
vertical_flip=False)

# Compute quantities required for featurewise normalization
# (std, mean, and principal components if ZCA whitening is applied).
datagen.fit(x_train2)

# Fit the model on the batches generated by datagen.flow().
model.fit_generator(datagen.flow(x_train2, y_train2, batch_size=128),
validation_data=(x_test2, y_test2),
epochs=10, verbose=1, workers=-1,
callbacks=callbacks)

# Score trained model.
scores = model.evaluate(x_test2, y_test2, verbose=1)
print('Test loss:', scores[0])
print('Test accuracy:', scores[1])
\end{lstlisting}


\begin{lstlisting}[language=Python,label={code:squeezenet},caption={SqueezeNet Code}]
#conv 1
conv1 = Convolution2D(96, 3, 3, activation='relu', init='glorot_uniform',subsample=(2,2),border_mode='valid')(input_layer)

#maxpool 1
maxpool1 = MaxPooling2D(pool_size=(2,2))(conv1)

#fire 1
fire2_squeeze = Convolution2D(16, 1, 1, activation='relu', init='glorot_uniform',border_mode='same')(maxpool1)
fire2_expand1 = Convolution2D(64, 1, 1, activation='relu', init='glorot_uniform',border_mode='same')(fire2_squeeze)
fire2_expand2 = Convolution2D(64, 3, 3, activation='relu', init='glorot_uniform',border_mode='same')(fire2_squeeze)
merge1 = merge(inputs=[fire2_expand1, fire2_expand2], mode="concat", concat_axis=1)
fire2 = Activation("linear")(merge1)

#fire 2
fire3_squeeze = Convolution2D(16, 1, 1, activation='relu', init='glorot_uniform',border_mode='same')(fire2)
fire3_expand1 = Convolution2D(64, 1, 1, activation='relu', init='glorot_uniform',border_mode='same')(fire3_squeeze)
fire3_expand2 = Convolution2D(64, 3, 3, activation='relu', init='glorot_uniform',border_mode='same')(fire3_squeeze)
merge2 = merge(inputs=[fire3_expand1, fire3_expand2], mode="concat", concat_axis=1)
fire3 = Activation("linear")(merge2)

#fire 3
fire4_squeeze = Convolution2D(32, 1, 1, activation='relu', init='glorot_uniform',border_mode='same')(fire3)
fire4_expand1 = Convolution2D(128, 1, 1, activation='relu', init='glorot_uniform',border_mode='same')(fire4_squeeze)
fire4_expand2 = Convolution2D(128, 3, 3, activation='relu', init='glorot_uniform',border_mode='same')(fire4_squeeze)
merge3 = merge(inputs=[fire4_expand1, fire4_expand2], mode="concat", concat_axis=1)
fire4 = Activation("linear")(merge3)

#maxpool 4
maxpool4 = MaxPooling2D((2,2))(fire4)

#fire 5
fire5_squeeze = Convolution2D(32, 1, 1, activation='relu', init='glorot_uniform',border_mode='same')(maxpool4)
fire5_expand1 = Convolution2D(128, 1, 1, activation='relu', init='glorot_uniform',border_mode='same')(fire5_squeeze)
fire5_expand2 = Convolution2D(128, 3, 3, activation='relu', init='glorot_uniform',border_mode='same')(fire5_squeeze)
merge5 = merge(inputs=[fire5_expand1, fire5_expand2], mode="concat", concat_axis=1)
fire5 = Activation("linear")(merge5)

#fire 6
fire6_squeeze = Convolution2D(48, 1, 1, activation='relu', init='glorot_uniform',border_mode='same')(fire5)
fire6_expand1 = Convolution2D(192, 1, 1, activation='relu', init='glorot_uniform',border_mode='same')(fire6_squeeze)
fire6_expand2 = Convolution2D(192, 3, 3, activation='relu', init='glorot_uniform',border_mode='same')(fire6_squeeze)
merge6 = merge(inputs=[fire6_expand1, fire6_expand2], mode="concat", concat_axis=1)
fire6 = Activation("linear")(merge6)

#fire 7
fire7_squeeze = Convolution2D(48, 1, 1, activation='relu', init='glorot_uniform',border_mode='same')(fire6)
fire7_expand1 = Convolution2D(192, 1, 1, activation='relu', init='glorot_uniform',border_mode='same')(fire7_squeeze)
fire7_expand2 = Convolution2D(192, 3, 3, activation='relu', init='glorot_uniform',border_mode='same')(fire7_squeeze)
merge7 = merge(inputs=[fire7_expand1, fire7_expand2], mode="concat", concat_axis=1)
fire7 =Activation("linear")(merge7)

#fire 8
fire8_squeeze = Convolution2D(64, 1, 1, activation='relu', init='glorot_uniform',border_mode='same')(fire7)
fire8_expand1 = Convolution2D(256, 1, 1, activation='relu', init='glorot_uniform',border_mode='same')(fire8_squeeze)
fire8_expand2 = Convolution2D(256, 3, 3, activation='relu', init='glorot_uniform',border_mode='same')(fire8_squeeze)
merge8 = merge(inputs=[fire8_expand1, fire8_expand2], mode="concat", concat_axis=1)
fire8 = Activation("linear")(merge8)

#maxpool 8
maxpool8 = MaxPooling2D((2,2))(fire8)

#fire 9
fire9_squeeze = Convolution2D(64, 1, 1, activation='relu', init='glorot_uniform',border_mode='same')(maxpool8)
fire9_expand1 = Convolution2D(256, 1, 1, activation='relu', init='glorot_uniform',border_mode='same')(fire9_squeeze)
fire9_expand2 = Convolution2D(256, 3, 3, activation='relu', init='glorot_uniform',border_mode='same')(fire9_squeeze)
merge8 = merge(inputs=[fire9_expand1, fire9_expand2], mode="concat", concat_axis=1)
fire9 = Activation("linear")(merge8)
fire9_dropout = Dropout(0.5)(fire9)

#conv 10
conv10 = Convolution2D(10, 1, 1, init='glorot_uniform',border_mode='valid')(fire9_dropout)

#avgpool 1
avgpool10 = AveragePooling2D((13,13), strides=(1,1), border_mode='same')(conv10)

flatten = Flatten()(avgpool10)

softmax = Dense(10, activation="softmax")(flatten)

model = Model(input=input_layer, output=softmax)

model.summary()

model.compile(optimizer='adadelta', loss="categorical_crossentropy", metrics=["accuracy"])
model.fit(x_train, y_train, batch_size=128, epochs=10, validation_data=(x_test, y_test))

score = model.evaluate(x_test, y_test, verbose=1)
print('Test score:', score[0])
print('Test accuracy:', score[1])
\end{lstlisting}

\begin{lstlisting}[language=Python,label={code:mnist},caption={Code for loading MNIST dataset}]
from keras.datasets import mnist

(x_train, y_train), (x_test, y_test) = mnist.load_data()
print("x_train shape: {}".format(x_train.shape))
print("y_train shape: {}".format(y_train.shape))
x_train = x_train.reshape(x_train.shape[0], 28, 28, 1)
x_test = x_test.reshape(x_test.shape[0], 28, 28, 1)

y_train = np_utils.to_categorical(y_train, 10)
y_test = np_utils.to_categorical(y_test, 10)
\end{lstlisting}

\begin{lstlisting}[language=Python,label={code:fashionmnist},caption={Code for loading Fashion MNIST dataset}]
from keras.datasets import fashion_mnist

(x_train, y_train), (x_test, y_test) = fashion_mnist.load_data()
print("x_train shape: {}".format(x_train.shape))
print("y_train shape: {}".format(y_train.shape))
x_train = x_train.reshape(x_train.shape[0], 28, 28, 1)
x_test = x_test.reshape(x_test.shape[0], 28, 28, 1)

y_train = np_utils.to_categorical(y_train, 10)
y_test = np_utils.to_categorical(y_test, 10)
\end{lstlisting}

\begin{lstlisting}[language=Python,label={code:cifar10},caption={Code for loading CIFAR-10 dataset}]
from keras.datasets import cifar10

(x_train, y_train), (x_test, y_test) = cifar10.load_data()
y_train = keras.utils.to_categorical(y_train, 10)
y_test = keras.utils.to_categorical(y_test, 10)
x_train = x_train.astype('float32')
x_test = x_test.astype('float32')
\end{lstlisting}



\begin{lstlisting}[language=Python,label={code:SVHN},caption={Code for loading SVHN dataset}]
def load_data(path):
""" Helper function for loading a MAT-File"""
data = loadmat(path)
return data['X'], data['y']

X_train, y_train = load_data('train_32x32.mat')
X_test, y_test = load_data('test_32x32.mat')
X_extra, y_extra = load_data('extra_32x32.mat')


\end{lstlisting}

