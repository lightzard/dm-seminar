\section{Context \& Motivation}
Machine learning has become an integral part of today's technology. It has a lot of applications in our daily lives, for example a recommender system or a prediction models. One of the machine learning technique that we often see nowadays is deep learning. It is the latest topics in the machine learning research which is proven to do well in solving complex classification task such as image and speech recognition \cite{krizhevsky2012imagenet}. The ability to discover intricate structure makes it strong tools for high dimensional data processing such as image data. In this research, we are primarily interested to explore deep learning on image datasets.

The challenging part for doing a deep learning is deciding the architecture to use for a given image datasets. There is no exact guidelines on designing a deep learning architecture. Several architecture has been proposed for a specific datasets, however we rarely see the performance comparison of the proposed design for the same datasets. Such comparison will tells us in what cases does an architecture performs better compared to the other. A comparison also tells us whether there exists an architecture that generally works well for a general image dataset or what kind of datasets criteria that works well in an architecture.

To solve this problem, we propose a benchmarking studies of multiple deep learning architecture on many image datasets. The goal of the research is to have a benchmark analysis of various deep learning architecture performance on multiple image datasets.   \\



%This is a demo \LaTeX template you can use for your TU/e Master of Science thesis. It can of course also be used for theses at other universities (be sure to use a high quality picture!!!) and other types of theses.

%In this particular template special care has been taken to position the thesis title correctly on the front page for the TU/e see-through box in the default cover. However, please check (and double check!) if this is still correct before you send your thesis to the repro shop and have copies printed. Additionally, we defined the command \textbackslash todo\{\} that produces a \todo{The ToDo note} ToDo note in the margin (and introduces some white space in the text unfortunately).

%Good luck with your thesis and be sure to share the end result with us on \url{joosbuijs.wordpress.com}!

%\vspace{10pt}

%Joos Buijs

%Eindhoven,

%March 2013.


\section{Research Question}
The works tries to answer the following research question:
"How does the performance comparison of deep learning architecture model looks like for a given image classification dataset?"
In answering this research question, the approach that we try to use is to implement the state-of-the art and widely-used deep learning architecture in various datasets and analyze its performance. Some results from previous related research paper will also be used for benchmarking. There are also some sub-questions to solve the main research questions, which are:
\begin{enumerate}
	\item Which architecture that works bests for a given datasets?
	\item What kind of datasets characteristics that makes a deep learning architecture works well?
	\item Is there any architecture that generally works well for image classification?	
\end{enumerate}
%that yields a good performance1. 

